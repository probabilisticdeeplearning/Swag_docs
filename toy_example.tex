\section{Toy example}

\subsection{Data}

Despite being a toy example, the dimensionality needs to be high. Consider simulated data, with $n$ normally distributed observations with mean parameter $\bm{\theta}$ from a similarly normal prior with known variance:

\begin{align}
  &\text{Likelihood:} &&\bm{x} | \bm{\theta} \sim \mathcal{N}(\bm{x}; \bm{\theta}, \Sigma_{\bm{x}})  \nonumber \\
  &\text{Prior:} &&\bm{\theta} \sim \mathcal{N}(\bm{\theta}; \bm{\theta}_0, \Sigma_{\bm{\theta}}),
\end{align}

where

\begin{equation}
  \mathcal{N} \left( \bm{x}; \bm{\mu}, \Sigma \right) = \frac{ \exp \left( -\frac{ 1 }{ 2 } ( \bm {x} - \bm{\mu} )^{ \mathrm {T} }{ \Sigma }^{-1} ( \bm{x} - \bm{\mu} ) \right)} { \sqrt{ ( 2\pi )^{k} | \Sigma | } }.
\end{equation}

and the corresponding negative log. likelihood:

\begin{equation}
  -\log \mathcal{N} \left( \bm{x}; \bm{\mu}, \Sigma \right) = \frac{ 1 }{ 2 } ( \bm {x} - \bm{\mu} )^{ \mathrm {T} }{ \Sigma }^{-1} ( \bm{x} - \bm{\mu} ) + \frac{k}{2} \log 2\pi  + \frac{1}{2} \log | \Sigma | .
\end{equation}
Under these assumptions the likelihood and prior conjugate and the posterior distribution is also normal:

\begin{align}
  \bm{\theta}|\bm{x} &\sim \mathcal{N}(\bm{\theta}; \bm{\theta}', \Sigma_{\bm{\theta}}') \nonumber,
\end{align}
where

\begin{align}
  \bm{\theta}' &= \left( \Sigma_{\bm{\theta}}^{-1} + n \Sigma_x^{-1}\right)^{-1} \left( \Sigma_{\bm{\theta}}^{-1} \bm{\theta}_{0} + \Sigma_x^{-1} \sum_{i=1}^{n} \bm{x}_{i} \right) \nonumber \\
  \Sigma_{\bm{\theta}}' &=\left( \Sigma_{\bm{\theta}}^{-1} + n \Sigma_x^{-1}\right)^{-1}
  \label{eq:posterior_update}
\end{align}
A sample $\{\bm{x}_1, \dots, \bm{x}_B \}$ with batch size $B$ is created by first sampling a $\theta$ from the prior and then use that $\theta$ when sampling from $ \bm{x} \sim \theta $.
Then the true posterior will be determined by updating the prior (eq \eqref{eq:posterior_update}) after each batch and consider the last posterior as the current prior.

Perhaps I should let $\Sigma_x$ be a parameter as well? But if we want a normal posterior then it needs to be known.



\subsection{SGD}

If we instead wish to retrieve the posterior parameters with stochastic gradient descent we construct a loss function based on the probability of the observations given a set of parameters:
\begin{equation}
  \mathcal{L} = n \log |\Sigma_x| + \displaystyle{\sum_{i=1}^n} (\bm{x}_i - \bm{\theta})^T \Sigma_x^{-1} (\bm{x}_i - \bm{\theta}) \propto - \log p(\{\bm{x}_1, \dots, \bm{x}_n\}| \bm{\theta}, \Sigma_x)
\end{equation}
Or is it better to base it on a batch updated posterior, $-\log p(\bm{\theta}| \bm{x})$? In the swag article they seem to use the posterior~\cite{swag}.

Update: I am now using the posterior.

\subsection{Learning}

I use the posterior as my loss function, in practice I sum the negative logarithm of the likelihood and the posterior:

\begin{align}
  \mathcal{L} \propto \log p( \bm{\theta} | \bm{x} )- \displaystyle{\sum_{i=1}^n} \log p( \bm{x}_i | \bm{\theta} ) - n \log p( \bm{\theta} ) \nonumber \\ 
\end{align}
and use the \texttt{pytorch} implementations of the distributions to calculate the log probability. The code for algorithm \ref{alg:swag_gaussian} is in the
\href{https://github.com/probabilisticdeeplearning/swa_gaussian/tree/toy_example}{\texttt{repo}} (branch \texttt{toy\_example})

\begin{algorithm}
  \SetKwInOut{Input}{Input}
  \SetKwInOut{Output}{Output}
  \SetKw{KwBy}{by}

  \underline{function Euclid} $(a,b)$\;
  \Input{$\bm{\theta}_0, \Sigma_{\bm{\theta}}, \Sigma_{\bm{x}}$}
  \Output{$\{\bm{\hat{\theta}}\}$}
  $\{\bm{\hat{\theta}}\} = \emptyset$\\
  \For{epoch in $1, \dots, T$}
  {
    \For{batch in $1, \dots, n/B$}
    {
      l = 0
      \For{i in $1, \dots, B$}
      {
        sample $\bm{\theta} \sim \bm{\theta} | \bm{\theta}_0$\\
        sample $\bm{x} \sim \bm{x}_i | \bm{\theta}$\\
        $l \to l + \mathcal{L}(\bm{x}_i, \bm{\hat{\theta}})$\\
      }
      $\bm{\hat{\theta}} \to sgd(\bm{\hat{\theta}})$\\
      update\_posterior\\
      $\{\bm{\hat{\theta}}\} \to \{\bm{\hat{\theta}}\} \cup \bm{\hat{\theta}}$
    }
  }
  \caption{Swag, Normal likelihood, normal prior}
  \label{alg:swag_gaussian}
\end{algorithm}
